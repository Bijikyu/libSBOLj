\section*{Getting Started with libSBOLj 2.0}

This beginner's guide introduces the creation and use of \lstinline+TopLevel+ SBOL 2.0 objects using libSBOLj 2.0.  This guide is not meant to be a comprehensive documentation of the library but rather a simple introduction to the basic methods available.  In particular, this document demonstrates
the usage of the library's API through the creation of \lstinline+ComponentDefinition+ and \lstinline+Sequence+ entities for part of a genetic inverter.  A similar approach can be taken for the other SBOL 2.0 \lstinline+TopLevel+ objects, such as \lstinline+ModuleDefinition+, \lstinline+Model+, \lstinline+Collection+, and \lstinline+GenericTopLevel+.  Readers who are not familiar with Java are recommended to use other resources to learn about its basics. 

\subsection*{Creating an SBOL Document}

An \lstinline+SBOLDocument+ object is a container for \lstinline+TopLevel+ data objects.  Therefore, the first step is to create an \lstinline+SBOLDocument+ to put our \lstinline+TopLevel+ objects into.
This can be done by calling the \lstinline+SBOLDocument+ constructor as shown below. 

\begin{minipage}{0.95\textwidth} 
\begin{lstlisting}[basicstyle=\footnotesize\ttfamily]
  String prURI="http://partsregistry.org"; 
  SBOLDocument document = new SBOLDocument();		
  document.setDefaultURIprefix(prURI);
  document.setTypesInURIs(true);
  document.setComplete(true);
  document.setCreateDefaults(true);
\end{lstlisting}
\end{minipage}

The method \lstinline+setDefaultURIprefix+ sets the
default URI prefix to the value defined by the \lstinline+prURI+
string. All data objects created following this statement carry this
default URI prefix. If we want to allow all top-level identity URIs to
include their types, then we can call \lstinline+document.setTypesInURIs(true)+.
The \lstinline+document.setComplete(true)+ statement sets the ``complete'' flag
to \lstinline+true+ for the given \lstinline+document+. It means that
any URIs that cannot be dereferenced to a valid object in the current \lstinline+SBOLDocument+ object cause an exception to be thrown.  Finally, the \lstinline+document.setCreateDefaults(true)+ statement sets the ``createDefaults'' flag to \lstinline+true+ for the given \lstinline+document+.  It means that when an object that must reference a \lstinline+ComponentInstance+ object cannot find the component, but it can find a \lstinline+ComponentDefinition+ object with the same \lstinline+displayId+ that it creates a default \lstinline+ComponentInstance+ object with this \lstinline+displayId+ and references it.

\subsection*{Creating SBOL Data Objects}

The next step is to create \lstinline+TopLevel+ objects.  All create methods take as parameters an optional URI prefix \lstinline+String+, a display id \lstinline+String+, and an optional version \lstinline+String+.  Each create method also has parameters for each required element of a \lstinline+TopLevel+ object.  For a \lstinline+ComponentDefinition+, the only required element is a set of type URIs.  The code below creates \lstinline+ComponentDefinition+ objects for a promoter repressed by TetR, a coding sequence for the LacI repressor protein, and a cassette that includes them both.  The \lstinline+types+ property creates a set of URIs to specify the biochemical or physical entity for each \lstinline+ComponentDefinition+, which, in this case, is a double-stranded DNA. The first URI \lstinline+ComponentDefinition.DNA+ is a BioPAX ontology term defined as a constant in the \lstinline+ComponentDefinition+ class. The second URI is a \emph{Chemical Entities of Biological Interest} (ChEBI) term specified by the given URI.  Since only the display id is provided to the create functions below, the default URI prefix is used and no version is set.

\begin{minipage}{0.95\textwidth} 
\begin{lstlisting}[basicstyle=\footnotesize\ttfamily]
  HashSet<URI> types =  new HashSet<URI>(Arrays.asList(ComponentDefinition.DNA,
                          URI.create("http://identifiers.org/chebi/CHEBI:4705")));
  ComponentDefinition TetR_promoter = document.createComponentDefinition(
                                   "BBa_R0040",
                                   types);
  ComponentDefinition LacI_repressor = document.createComponentDefinition(
				"BBa_C0012", 
				types);
  ComponentDefinition pIKELeftCassette = document.createComponentDefinition(
				"pIKELeftCassette", 
				types);
\end{lstlisting}
\end{minipage}

The create sequence methods take an optional URI prefix \lstinline+String+, a display id \lstinline+String+, an optional version \lstinline+String+, a \lstinline+String+ providing the elements of the sequence, and a URI specifying the encoding used by the sequence.  The code below
creates sequences for the TetR repressible promoter and LacI coding sequence.

\begin{minipage}{0.95\textwidth} 
\begin{lstlisting}[basicstyle=\footnotesize\ttfamily]
  Sequence seq_187 = document.createSequence(
                     "seq_187",
                     "tccctatcagtgatagagattgacatccctatcagtgatagagatactgagcac",
                     Sequence.IUPAC_DNA
                     );
  String element2 = "atggtgaatgtgaaaccagtaacgttatacgatgtcgcagagtatgccggtgtc"
             + "tcttatcagaccgtttcccgcgtggtgaaccaggccagccacgtttctgcgaaaacgcggga"
             + "aaaagtggaagcggcgatggcggagctgaattacattcccaaccgcgtggcacaacaactgg"
             + "cgggcaaacagtcgttgctgattggcgttgccacctccagtctggccctgcacgcgccgtcg"
             + "caaattgtcgcggcgattaaatctcgcgccgatcaactgggtgccagcgtggtggtgtcgat"
             + "ggtagaacgaagcggcgtcgaagcctgtaaagcggcggtgcacaatcttctcgcgcaacgcg"
             + "tcagtgggctgatcattaactatccgctggatgaccaggatgccattgctgtggaagctgcc"
             + "tgcactaatgttccggcgttatttcttgatgtctctgaccagacacccatcaacagtattat"
             + "tttctcccatgaagacggtacgcgactgggcgtggagcatctggtcgcattgggtcaccagc"
             + "aaatcgcgctgttagcgggcccattaagttctgtctcggcgcgtctgcgtctggctggctgg"
             + "cataaatatctcactcgcaatcaaattcagccgatagcggaacgggaaggcgactggagtgc"
             + "catgtccggttttcaacaaaccatgcaaatgctgaatgagggcatcgttcccactgcgatgc"
             + "tggttgccaacgatcagatggcgctgggcgcaatgcgcgccattaccgagtccgggctgcgc"
             + "gttggtgcggatatctcggtagtgggatacgacgataccgaagacagctcatgttatatccc"
             + "gccgttaaccaccatcaaacaggattttcgcctgctggggcaaaccagcgtggaccgcttgc"
             + "tgcaactctctcagggccaggcggtgaagggcaatcagctgttgcccgtctcactggtgaaa"
             + "agaaaaaccaccctggcgcccaatacgcaaaccgcctctccccgcgcgttggccgattcatt"
             + "aatgcagctggcacgacaggtttcccgactggaaagcgggcaggctgcaaacgacgaaaact"
             + "acgctttagtagcttaataa";
  Sequence seq_153 = document.createSequence(
                     "seq_153",
                     element2,
                     Sequence.IUPAC_DNA
                     );
\end{lstlisting}
\end{minipage}


\subsection*{Setting and Editing Optional Fields}

Next, we set and edit the optional fields for the \lstinline+TopLevel+ objects. For any optional field that is not a set or list, the library provides methods to set its value, unset its value, and check its value. The only exceptions where these methods are not available are the following three fields in the \lstinline+Identified+ class: \lstinline+persistentIdentity+, \lstinline+displayId+, and
\lstinline+version+.  These fields cannot be edited, since they are crucial to maintaining
compliant SBOL objects (see Section~11.2 ``Compliant SBOL Objects'' of the~\href{http://sbolstandard.org/downloads/specification-data-model-2-0/}{Specification
  (Data Model 2.0)} for more details).  The example code below first sets the name and description of the
TetR repressible promoter and LacI repressor.  Next, it checks if the name is set for \lstinline+TetR_promoter+, then unsets it, and sets it to a new name.
 
\begin{minipage}{0.95\textwidth} 
\begin{lstlisting}[basicstyle=\footnotesize\ttfamily]
  TetR_promoter.setName("p tetR");
  LacI_repressor.setName("lacI");
  TetR_promoter.setDescription("TetR repressible promoter");
  LacI_repressor.setDescription("lacI repressor from E. coli (+LVA)");
  if (TetR_promoter.isSetName()) {
	TetR_promoter.unsetName();
        TetR_promoter.setName("p(tetR)");
  }
\end{lstlisting}
\end{minipage}

For an optional field that is either a list or a set, the library provides methods for adding, removing, and checking if an element is contained in the list or set.  The code below first adds roles to the \lstinline+TetR_promoter+ and \lstinline+LacI_repressor+ using constants in the
\lstinline+SequenceOntology+ class that indicates that they are a promoter and coding sequence respectively.  Next, it adds a second role to \lstinline+TetR_promoter+, checks if the role is contained in the set of roles, then removes the role.

\begin{minipage}{0.95\textwidth} 
\begin{lstlisting}[basicstyle=\footnotesize\ttfamily]
  TetR_promoter.addRole(SequenceOntology.PROMOTER);		
  LacI_repressor.addRole(SequenceOntology.CDS);
  URI TetR_promoter_role2 = URI.create("http://identifiers.org/so/SO:0000613"); 
  TetR_promoter.addRole(TetR_promoter_role2);
  if (TetR_promoter.containsRole(TetR_promoter_role2)) {
	TetR_promoter.removeRole(TetR_promoter_role2);
  }
\end{lstlisting}
\end{minipage}

Fields that are a list or set of objects also include operations to clear, get, and set them.  The example code below removes all roles at once by calling \lstinline+clearRoles()+, gets the set of roles for the \lstinline+TetR_promoter+, and checks if it is empty.   Finally, it sets the entire set of roles (replacing any existing set) by calling \lstinline+setRoles(Set<URI> roles)+.

\begin{minipage}{0.95\textwidth} 
\begin{lstlisting}[basicstyle=\footnotesize\ttfamily]
  TetR_promoter.clearRoles();
  if (!TetR_promoter.getRoles().isEmpty()) {
	System.out.println("TetR_promoter set is not empty");
  }
  TetR_promoter.setRoles(new HashSet<URI>(Arrays.asList(
 				SequenceOntology.PROMOTER)));
\end{lstlisting}
\end{minipage}


\subsection*{Creating and Editing References}

\lstinline+TopLevel+ objects can refer to other \lstinline+TopLevel+ objects.  For example, 
a \lstinline+ComponentDefinition+ object can refer to one or more sequences.  This reference is created by calling the \lstinline+addSequence(URI)+ method. Note that since the ``complete'' flag is set to \lstinline+true+, the library verifies that all objects referenced are present in this \lstinline+document+. Methods available for manipulating references are similar
to those for the optional fields. The code below adds one sequence for
each of the \lstinline+ComponentDefinition+ objects, and then deletes all references for \lstinline+pIKELeftCassette+.  Finally, if we include the last line, an exception is thrown because a sequence with the specified URI cannot be found.

\begin{minipage}{0.95\textwidth} 
\begin{lstlisting}[basicstyle=\footnotesize\ttfamily]
  TetR_promoter.addSequence(seq_187);		
  LacI_repressor.addSequence(seq_153);		
  pIKELeftCassette.addSequence(seq_187);
  pIKELeftCassette.clearSequences();
  //pIKELeftCassette.addSequence(
  //   URI.create("http://partsregistry.org/seq/partseq_154"));
\end{lstlisting}
\end{minipage}

\subsection*{Creating and Editing Child Objects}

The \lstinline+ModuleDefinition+ and \lstinline+ComponentDefinition+ objects have child objects that can be created and edited.  The code below creates a child object for \lstinline+pIKELeftCassette+.
Namely, it creates a child \lstinline+SequenceConstraint+ object that says the \lstinline+TetR_promoter+ precedes the \lstinline+LacI_repressor+.  A sequence constraint has a
``subject'' and an ``object'' reference that point to two different \lstinline+Component+ objects. They do not currently exist in our document and need to be created. With the ``createDefaults'' flag
set to \lstinline+true+, creation of these components is done automatically.  Each component created has the same \lstinline+displayId+ as its corresponding \lstinline+ComponentDefinition+
object.  They are added to \lstinline+pIKELeftCassette+'s set of components. 
This can be verified by calling the \lstinline+getComponent+ method.  Finally, if we include the last line, it causes an exception, since the component being removed is used by a sequence constraint. 

\begin{minipage}{0.95\textwidth} 
\begin{lstlisting}[basicstyle=\footnotesize\ttfamily]
  pIKELeftCassette.createSequenceConstraint(
         "pIKELeftCassette_sc",
         RestrictionType.PRECEDES,
         TetR_promoter.getDisplayId(), 
         LacI_repressor.getDisplayId()
         );
  if (pIKELeftCassette.getComponent("BBa_R0040")==null) {
	System.out.println("TetR_promoter component is missing");
  }
  if (pIKELeftCassette.getComponent("BBa_C0012")==null) {
	System.out.println("LacI_repressor component is missing");
  }
  //pIKELeftCassette.removeComponent(
  //                          pIKELeftCassette.getComponent("BBa_R0040"));
\end{lstlisting}
\end{minipage}

\subsection*{Copying Objects}

Finally, the library can make copies of \lstinline+TopLevel+ objects using the \lstinline+createCopy+ methods.  There are several variations of this method.  If only the object is provided, an identical copy is created.  Note that this will cause an exception if it is copied into the same document, since it will not have a unique identity.  Next, a new display id can be provided, which results in a copy with the display ids (and identities) updated accordingly.  Similarly, a version can be provided as well, which results in a copy with the version (and identities) updated accordingly.  Finally, a new URI prefix can also be provided, once again resulting in updated identities.  The example below makes a copy of the
\lstinline+TetR_promoter+ object by calling \lstinline+createCopy+ with a new display id, \lstinline+BBa_K137046+. The identity URI for \lstinline+TetR_promoter_copy+ is changed to include the new display id, and this change also percolates through the identity URIs for any descendent objects.  This example gives the new copy a new sequence, but it keeps all other properties the same. 

\begin{minipage}{0.95\textwidth} 
\begin{lstlisting}[basicstyle=\footnotesize\ttfamily]
  ComponentDefinition TetR_promoter_copy = 
      (ComponentDefinition)document.createCopy(TetR_promoter, "BBa_K137046");
  Sequence seq = document.createSequence(
 		"seq_K137046",
		"gtgctcagtatctctatcactgatagggatgtcaatctctatcactgatagggactctagtatat"
		+ "aaacgcagaaaggcccacccgaaggtgagccagtgtgactctagtagagagcgttcaccgaca"
		+ "aacaacagataaaacgaaaggc",
		Sequence.IUPAC_DNA
		);	
 TetR_promoter_copy.addSequence(seq);
\end{lstlisting}
\end{minipage}

The example used throughout this tutorial is listed as
``GettingStartedExample.java'' in the libSBOLj 2.0 examples
directory. The RDF serialization for this example is shown below.  

%% TODO: write/read

\begin{minipage}{0.95\textwidth} 
\lstsetsbol
\begin{lstlisting}
<?xml version="1.0" ?>
<rdf:RDF xmlns:rdf="http://www.w3.org/1999/02/22-rdf-syntax-ns#" xmlns:dcterms="http://purl.org/dc/terms/" xmlns:prov="http://www.w3.org/ns/prov#" xmlns:sbol="http://sbols.org/v2#">
  <sbol:ComponentDefinition rdf:about="http://partsregistry.org/BBa_K137046">
    <sbol:persistentIdentity rdf:resource="http://partsregistry.org/BBa_K137046"/>
    <sbol:displayId>BBa_K137046</sbol:displayId>
    <prov:wasDerivedFrom rdf:resource="http://partsregistry.org/cd/BBa_R0040"/>
    <dcterms:title>p(tetR)</dcterms:title>
    <dcterms:description>TetR repressible promoter</dcterms:description>
    <sbol:type rdf:resource="http://identifiers.org/chebi/CHEBI:4705"/>
    <sbol:type rdf:resource="http://www.biopax.org/release/biopax-level3.owl#DnaRegion"/>
    <sbol:role rdf:resource="http://identifiers.org/so/SO:0000167"/>
    <sbol:sequence rdf:resource="http://partsregistry.org/seq/partseq_187"/>
    <sbol:sequence rdf:resource="http://partsregistry.org/seq/seq_K137046"/>
  </sbol:ComponentDefinition>
  <sbol:ComponentDefinition rdf:about="http://partsregistry.org/cd/BBa_C0012">
    <sbol:persistentIdentity rdf:resource="http://partsregistry.org/cd/BBa_C0012"/>
    <sbol:displayId>BBa_C0012</sbol:displayId>
    <dcterms:title>lacI</dcterms:title>
    <dcterms:description>lacI repressor from E. coli (+LVA)</dcterms:description>
    <sbol:type rdf:resource="http://identifiers.org/chebi/CHEBI:4705"/>
    <sbol:type rdf:resource="http://www.biopax.org/release/biopax-level3.owl#DnaRegion"/>
    <sbol:role rdf:resource="http://identifiers.org/so/SO:0000316"/>
    <sbol:sequence rdf:resource="http://partsregistry.org/seq/partseq_153"/>
  </sbol:ComponentDefinition>
  <sbol:ComponentDefinition rdf:about="http://partsregistry.org/cd/BBa_R0040">
    <sbol:persistentIdentity rdf:resource="http://partsregistry.org/cd/BBa_R0040"/>
    <sbol:displayId>BBa_R0040</sbol:displayId>
    <dcterms:title>p(tetR)</dcterms:title>
    <dcterms:description>TetR repressible promoter</dcterms:description>
    <sbol:type rdf:resource="http://identifiers.org/chebi/CHEBI:4705"/>
    <sbol:type rdf:resource="http://www.biopax.org/release/biopax-level3.owl#DnaRegion"/>
    <sbol:role rdf:resource="http://identifiers.org/so/SO:0000167"/>
    <sbol:sequence rdf:resource="http://partsregistry.org/seq/partseq_187"/>
  </sbol:ComponentDefinition>
  <sbol:ComponentDefinition rdf:about="http://partsregistry.org/cd/pIKELeftCassette">
    <sbol:persistentIdentity rdf:resource="http://partsregistry.org/cd/pIKELeftCassette"/>
    <sbol:displayId>pIKELeftCassette</sbol:displayId>
    <sbol:type rdf:resource="http://identifiers.org/chebi/CHEBI:4705"/>
    <sbol:type rdf:resource="http://www.biopax.org/release/biopax-level3.owl#DnaRegion"/>
    <sbol:component>
      <sbol:Component rdf:about="http://partsregistry.org/cd/pIKELeftCassette/BBa_R0040">
        <sbol:persistentIdentity rdf:resource="http://partsregistry.org/cd/pIKELeftCassette/BBa_R0040"/>
        <sbol:displayId>BBa_R0040</sbol:displayId>
        <sbol:access rdf:resource="http://sbols.org/v2#public"/>
        <sbol:definition rdf:resource="http://partsregistry.org/cd/BBa_R0040"/>
      </sbol:Component>
    </sbol:component>
    <sbol:component>
      <sbol:Component rdf:about="http://partsregistry.org/cd/pIKELeftCassette/BBa_C0012">
        <sbol:persistentIdentity rdf:resource="http://partsregistry.org/cd/pIKELeftCassette/BBa_C0012"/>
        <sbol:displayId>BBa_C0012</sbol:displayId>
        <sbol:access rdf:resource="http://sbols.org/v2#public"/>
        <sbol:definition rdf:resource="http://partsregistry.org/cd/BBa_C0012"/>
      </sbol:Component>
    </sbol:component>
    <sbol:sequenceConstraint>
      <sbol:SequenceConstraint rdf:about="http://partsregistry.org/cd/pIKELeftCassette/pIKELeftCassette_sc">
        <sbol:persistentIdentity rdf:resource="http://partsregistry.org/cd/pIKELeftCassette/pIKELeftCassette_sc"/>
        <sbol:displayId>pIKELeftCassette_sc</sbol:displayId>
        <sbol:restriction rdf:resource="http://sbols.org/v2#precedes"/>
        <sbol:subject rdf:resource="http://partsregistry.org/cd/pIKELeftCassette/BBa_R0040"/>
        <sbol:object rdf:resource="http://partsregistry.org/cd/pIKELeftCassette/BBa_C0012"/>
      </sbol:SequenceConstraint>
    </sbol:sequenceConstraint>
  </sbol:ComponentDefinition>
\end{lstlisting}
\end{minipage}

\clearpage

\begin{minipage}{0.95\textwidth} 
\lstsetsbol
\begin{lstlisting}
  <sbol:Sequence rdf:about="http://partsregistry.org/seq/seq_K137046">
    <sbol:persistentIdentity rdf:resource="http://partsregistry.org/seq/seq_K137046"/>
    <sbol:displayId>seq_K137046</sbol:displayId>
    <sbol:elements>gtgctcagtatctctatcactgatagggatgtcaatctctatcactgatagggactctagtatataaacgcagaa
                   aggcccacccgaaggtgagccagtgtgactctagtagagagcgttcaccgacaaacaacagataaaacgaaaggc
    </sbol:elements>
    <sbol:encoding rdf:resource="http://www.chem.qmul.ac.uk/iubmb/misc/naseq.html"/>
  </sbol:Sequence>
  <sbol:Sequence rdf:about="http://partsregistry.org/seq/partseq_187">
    <sbol:persistentIdentity rdf:resource="http://partsregistry.org/seq/partseq_187"/>
    <sbol:displayId>partseq_187</sbol:displayId>
    <sbol:elements>tccctatcagtgatagagattgacatccctatcagtgatagagatactgagcac</sbol:elements>
    <sbol:encoding rdf:resource="http://www.chem.qmul.ac.uk/iubmb/misc/naseq.html"/>
  </sbol:Sequence>
  <sbol:Sequence rdf:about="http://partsregistry.org/seq/partseq_153">
    <sbol:persistentIdentity rdf:resource="http://partsregistry.org/seq/partseq_153"/>
    <sbol:displayId>partseq_153</sbol:displayId>
    <sbol:elements>atggtgaatgtgaaaccagtaacgttatacgatgtcgcagagtatgccggtgtctcttatcagaccgtttcccgcg
                   tggtgaaccaggccagccacgtttctgcgaaaacgcgggaaaaagtggaagcggcgatggcggagctgaattacat
                   tcccaaccgcgtggcacaacaactggcgggcaaacagtcgttgctgattggcgttgccacctccagtctggccctg
                   cacgcgccgtcgcaaattgtcgcggcgattaaatctcgcgccgatcaactgggtgccagcgtggtggtgtcgatgg
                   tagaacgaagcggcgtcgaagcctgtaaagcggcggtgcacaatcttctcgcgcaacgcgtcagtgggctgatcat
                   taactatccgctggatgaccaggatgccattgctgtggaagctgcctgcactaatgttccggcgttatttcttgat
                   gtctctgaccagacacccatcaacagtattattttctcccatgaagacggtacgcgactgggcgtggagcatctgg
                   tcgcattgggtcaccagcaaatcgcgctgttagcgggcccattaagttctgtctcggcgcgtctgcgtctggctgg
                   ctggcataaatatctcactcgcaatcaaattcagccgatagcggaacgggaaggcgactggagtgccatgtccggt
                   tttcaacaaaccatgcaaatgctgaatgagggcatcgttcccactgcgatgctggttgccaacgatcagatggcgc
                   tgggcgcaatgcgcgccattaccgagtccgggctgcgcgttggtgcggatatctcggtagtgggatacgacgatac
                   cgaagacagctcatgttatatcccgccgttaaccaccatcaaacaggattttcgcctgctggggcaaaccagcgtg
                   gaccgcttgctgcaactctctcagggccaggcggtgaagggcaatcagctgttgcccgtctcactggtgaaaagaa
                   aaaccaccctggcgcccaatacgcaaaccgcctctccccgcgcgttggccgattcattaatgcagctggcacgaca
                   ggtttcccgactggaaagcgggcaggctgcaaacgacgaaaactacgctttagtagcttaataa
    </sbol:elements>
    <sbol:encoding rdf:resource="http://www.chem.qmul.ac.uk/iubmb/misc/naseq.html"/>
  </sbol:Sequence>
</rdf:RDF>
\end{lstlisting}
\end{minipage}


